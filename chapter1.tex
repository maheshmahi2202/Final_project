
\newpage
\begin{center}
\section{INTRODUCTION}
\end{center}
\subsection{PREAMBLE}
\paragraph{}
With advances in technology, advents like Internet of Things (IoT) and Remote Sensing (RS) techniques are used in different areas of research for monitoring, collecting and analysis data from remote locations. Due to the enormous increase in global industrial output, rural to urban drift and the over-utilization of land and sea resources, the quality of water available to people has deteriorated immensely. The high usage of fertilizers in farms and also other chemicals in sectors such as mining and construction have contributed significantly to the overall reduction of water quality globally. 
\paragraph{}
Water is an essential need for human survival and therefore there must be mechanisms put in place to vigorously test the quality of water that is available for drinking in towns and cities with enough safe drinking water sources and as well as the rivers, creeks and shoreline that surround our towns and cities. The availability of good quality water is paramount in preventing outbreaks of water-borne diseases as well as improving the quality of life. Industrial waste outlets into river requires a frequent data collecting network for the water quality monitoring and IoT and RS can improve the existing traditional measurements. This project mainly deals with the general water monitoring parameters like temperature, pH, conductivity/salinity and turbidity where these values set a safety threshold and alert using IoT and RS technology.

\subsection{MOTIVATION OF THE PROJECT}
\paragraph{}
Rapid urbanization coupled with corrupted Municipal Corporations and poor infrastructure has led to the improper treatment of sewage water and industrial waste into lakes resulting in death of aquatic life and decline in fresh water for human consumption. Hence, there should be a need for efficient methodologies for monitoring the water quality to sustain the ecosystem.

\subsection{PROBLEM STATEMENT}
\paragraph{}
The real time monitoring of water quality is done by measuring temperature, turbidity, pH, conductivity in water using Arduino UNO, Wi-Fi module and different sensors in IoT Environment and notifying respective factories about their water quality. This procedure eliminates the manual work and transforms the evaluation into digital mode. Since the data is digitally processed, this project aims to eliminate the corruption from the factories and provide transparency to the environmentalists.  

\subsection{PRESENT SCENARIO}
\paragraph{}
Traditional methods of water quality involve, the manual collection of water sample at different factory locations, followed by laboratory analytical techniques in order to determine the water quality. Such approaches take longer time and no longer to be considered efficient. Although the current methodologies analyse the physical, chemical and biological agents, it has several drawbacks: 
\begin{itemize}
\item Poor spatio temporal coverage.
\item It is labour intensive and high cost (labor, operation; and equipment).
\item The lack of real time water quality information to enable critical decisions for public health protection. Therefore, there is a need for continuous online water quality monitoring.
\end{itemize}
\subsection{LITERATURE SURVEY – PROPOSED SYSTEM}
\paragraph{}
By focusing on the above issues we have developed a low cost system for real time monitoring of the water quality in connected environment. In our design Arduino UNO is used as a core controller. The collected sensor values are routed via the Wi-Fi module and uploaded to the cloud for storage and analysis. The sensor data can be viewed on the cloud using a special IP address. Additionally, the Wi-Fi module uses an SMTP protocol for viewing the data on mobile.
\begin{itemize}
\item Water quality can be monitored on real time basis [1]. From this we get an idea of how the water quality can be monitored and store the sensor values, also on how using internet the results of the tested water could be viewed and further actions could be taken up.
\item A research demonstrates a smart water quality monitoring system [2]. The temperature relation with pH and conductivity were also observed for all the water samples. GSM technology has been successfully implemented to send alarm based on reference parameter to the ultimate user for immediate action to ensure water quality. Additionally, the parameter references obtained from all the different water sources will be used to build classifiers which will be used to perform automated water analysis in the form of Neural Network Analysis.
\item Based on impedance measurements [3], this paper presents a portable sensor implemented as an electronic embedded system featuring disposable measurement cells, which is suitable of measuring bacterial concentration in water samples. We got to know that different sensors available in the market [3] could be used to test different parameters of the water quality. Our testing of water quality involves parameters like temperature, pH, conductivity and turbidity from the simulated values generated through software at the IoT core.
\item Especially students, practitioners and researchers who are interested in understanding and contributing to the ongoing merge [4] of the physical world of things and the Internet. We learn that the sensors and physical elements like water could be combined with internet for its testing. This reduces a lot of manual work involved in traditional system of water testing. 
\item As a pre-requisite, we must learn the essential basics about cloud computing and its applications [5]. We must learn that using cloud would be better than using the local server as the storage space in could is vast and data could be fetched from cloud whenever necessary.

\end{itemize}
